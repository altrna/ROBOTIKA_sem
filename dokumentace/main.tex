\documentclass[journal,twoside,web]{ieeecolor}
\usepackage{generic}
\usepackage{cite}
\usepackage{amsmath,amssymb,amsfonts}
\usepackage{algorithmic}
\usepackage{graphicx}
\usepackage{textcomp}
\usepackage{subfigure}
\usepackage{float}
\usepackage{hyperref}
\usepackage{lipsum} 
%\usepackage[english, czech]{babel}
%\usepackage[useregional]{datetime2}
%\DTMsetdatestyle{czech}
\usepackage[ddmmyyyy]{datetime}
\def\BibTeX{{\rm B\kern-.05em{\sc i\kern-.025em b}\kern-.08em
    T\kern-.1667em\lower.7ex\hbox{E}\kern-.125emX}}
\markboth{ČESKÉ VYSOKÉ UČENÍ TECHNICKÉ V PRAZE, FAKULTA ELEKTROTECHNICKÁ, KATEDRA KYBERNETIKY, \today}
{ČESKÉ VYSOKÉ UČENÍ TECHNICKÉ V PRAZE, FAKULTA ELEKTROTECHNICKÁ, KATEDRA KYBERNETIKY, \today}


\hypersetup{
    colorlinks=true,
    linkcolor=blue,
    filecolor=magenta,      
    urlcolor=cyan,
    pdftitle={Semestrální úloha z předmětu Robotika},
    pdfpagemode=FullScreen,
    }
\begin{document}
\title{Parkování robota v garáži}
\author{Aleš Trna, Minh Hoang Tran \\ \begin{center}
    \today
\end{center}
\thanks{}}

\maketitle

\begin{abstract}
    Předmětem této práce je popis řešení závěrečné semestrální úlohy z předmětu Robotika.
\end{abstract}

\begin{IEEEkeywords}
    Robot, manipulátor, 
    \end{IEEEkeywords}

\section{Zadání úlohy}
    V pracovním prostoru robota CRS97 jsou rozmístěny kostky o konstantní velikosti označené značkami typu \textit{Aruco}.
    Robot je umístěn v kleci, která vymezuje jeho pracovní prostor. Ke kleci je na konstantním místě připevněna kamera, která zabírá
    stále stejnou scénu, ale ze zadání neznáme její přesnou pozici. Naším úkolem je roztřídit Kostky s \textit{Aruco} značkami tak, aby
    kostky označené stejnými značkami byly uloženy na stejném místě.

\section{Měření a kalibrace kamery}
\subsection{Kalibrace kamery}
    Jako první musíme provést kalibraci kamery, Díky které 
\subsection{Hand to Eye kalibrace}
\end{document}
